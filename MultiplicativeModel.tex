\chapter{The Multiplicative Model}\label{Chapter:MultiplicativeModel}

The Multiplicative Model is one of the infinitely many ways to build stochastic descriptions for SAR data.
Among its advantages we would like to mention that it can be considered an \textit{ab initio} model, and that it leads to expressive and tractable descriptions of the data.

Let us recall that the basic model for multilook intensity data is the $\Gamma(\sigma^2,L)$ law whose density is
\begin{equation}
f_Z(z;L,\sigma^2) = \frac{L^L}{\sigma^{2L}\Gamma(L)} z^{L-1} 
	\exp\big\{ -L z / \sigma^2
	\big\}.
\end{equation}
As previously said, the Gamma distribution is scale-invariant, so we may pose this model as the product between the constant backscatter $X=\sigma^2$ and the multilook speckle $Y\sim\Gamma(1,L)$.

But, are there situations were we cannot assume a constant backscatter?
Yes, there are.

A constant backscatter results from infinitely many elementary backscatterers, i.e.\ from the assumption that $N\to\infty$ in~\eqref{Eq:ComplexBackscatter} (page~\pageref{Eq:ComplexBackscatter}).
Such assumption makes the particular choice of the sensed area irrelevant.
But this may not be the case always.

The advent of higher resolution sensors makes this hypothesis unsuitable in areas where the elementary backscatterers are of the order of the wavelength.
